% Include this file in your preamble!

% It defines handy macros for typesetting CCS claims:

% \edge P a Q prints an a-edge from P to Q. If \useEdgeOther has been used in the same environment before this command, the arrow is marked with a prime (')
\newcommand{\edge}[3]{#1 \stackrel{#2}{\longrightarrow} {#3}}
% Like \edge, but ignores \useEdgeOther
\newcommand{\edgeNormal}[3]{#1 \stackrel{#2}{\longrightarrow} {#3}}
% Like \edge, but always appends a prime (') to the arrow.
\newcommand{\edgeOther}[3]{#1 \stackrel{#2}{\longrightarrow } ' {#3}}
% Requests, that edge arrows are to appended a prime (')
\newcommand{\useEdgeOther}[1] {\renewcommand{\edge}{\edgeOther} {#1}}

% Nicely prints the name CCS_0
\newcommand{\CCSZero}{$CCS_0$ {}}
% Nicely prints the name CCS_0^ \omega
\newcommand{\CCSZeroW}{$CCS_0^\omega${}}
% Nicely prints the name CCS
\newcommand{\CCS}{$CCS${} }
% Nicely prints the name CCS_{vp}
\newcommand{\CCSvp}{$CCS_{vp}$ {}}


\usepackage {proof} % Used for inference rules!

% A nicer version of \infer from th proof package:
\newcommand{\inference}[3][]{\infer [\scriptstyle\mathrm {#1}] {#3} {#2}}

% The following rules simplify the type setting of inference trees:

\newcommand{\prefix}[2]
{\inference[prefix]{}{\edge {#1.#2} {#1} #2}}

\newcommand{\choicel}[5]
{\inference[choice{\_}l]{#1}{\edge {#2 + #3} {#4} {#5}}}
\newcommand{\choicer}[5]
{\inference[choice{\_}r]{#1}{\edge {#2 + #3} {#4} {#5}}}

\newcommand{\rec}[5]
{\inference[rec] {\Gamma(#1) = {#2} &	{#3}}{\edge {#1} {#4} {#5}} 	
 }

\newcommand{\parl}[5]
{\inference[par{\_}l]{#1}{\edge {{#2} | {#3}} {#4} {{#5} | {#3}}}}

\newcommand{\parr}[5]
{\inference[par{\_}r]{#1}{\edge {{#2} | {#3}} {#4} {{#2} | {#5}}}}


\newcommand{\sync}[6]
{\inference[sync]{#1 & #2}{\edge {{#3} | {#4}} \tau {{#5} | {#6}}}}

\newcommand{\res}[5]
{\inference[res]{#1 & #2 \notin #3}{\edge {#4 \setminus #3} {#2} {#5 \setminus #3}}}

\newcommand{\prefixvp}[2]
{\inference[prefix]{#1 \in Act}{\edge{#1.#2}{#1}{#2}}}

