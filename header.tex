\usepackage{graphicx} % used for the Programma logo in the title.
\usepackage{datetime} % used to retrieve the current month and year.
\usepackage{titling} % used for creating the title.

\usepackage{amssymb} % used for mathematical symbols
\usepackage{amsmath} % used for math
\usepackage{amsfonts} % used for mathematical fonts

\usepackage{textcomp} % used to load fonts that contain special symbols.
\usepackage{lmodern} % used because it is supposed to look better.
\usepackage{tabu} % For sized, content wrapping tables.
\usepackage[T1]{fontenc} % used because it is recommended for PDF viewers being able to process strange symbols in the right manner.

\usepackage{color} % used for coloring.
\usepackage{fontspec}
\usepackage{courier} % used for something that I forgot.

\usepackage{rotating} % used to rotate large inference trees.

\usepackage{import} %for inputting images in the 'images' directory.

\usepackage{stmaryrd} %used for semantic brackets: \llbracket \rrbracket

\usepackage{fancyhdr} %For footer/header

% Macro to include a pdf file that has been generated with Inkscape (PDF+LaTeX!!!) from the file images/#1.svg
% #2 is display width.
\newcommand {\svg}[2]{\def\svgwidth{#2}
\vtop{%
  \vskip0pt
  \hbox{
    \import{images/}{#1.pdf_tex}
  }
}}

\def\labelitemii{$\circ$} %Makes the second level bullet sign a circle.
\renewcommand{\ttdefault}{lmtt}  % Select a latin modern font as the standard typewriter font!

% Determine character spacing:
\setlength{\parindent}{0 cm}
\setlength{\parskip}{0.2 cm}



%Macro for TODOs:
\newcommand{\todo}[1]{\textcolor{red}  {\textbf{TODO: #1}}}


% For the end of proofs:
\newcommand{\qed}{\hfill $\blacksquare$}


%This macro is used to split text into columns:
\newenvironment{column}[1]
{\begin{minipage}[t]{#1\linewidth}}{\end{minipage} \bigskip\bigskip}
\newcommand{\nextColumn}[1]{\end{minipage}\begin{minipage}[t]{#1\linewidth}}


% No idea what these do:
\usepackage[small,compact]{titlesec}
\RequirePackage{etoolbox}
\RequirePackage{environ}

% Some ornaments:
\newcommand{\pgbreakhere}{\vskip0pt plus3cm\penalty-250\vskip0pt plus-3cm}

\newcommand{\raisedrule}[2][0em]{\leavevmode\leaders\hbox{\rule[#1]{1pt}{#2}}\hfill\kern0pt}

\newcommand{\shiftup}{\vskip-12mm ~\\}

% Prints a nice document head, with given title.
\newcommand{\Head}[1]{
   \begin{center}
   				\renewcommand*\familydefault{\sfdefault}
   				\normalfont\LARGE 
                 Nebenl\"aufige Programmierung (SS 2013)\\[1ex]
                 \raisedrule[5pt]{1pt}\hspace{2em}
                 #1
                 \hspace{2em}\raisedrule[5pt]{1pt}\\[1ex]
   \end{center}
  \smallskip
}
      
\renewcommand{\labelenumi}{(\alph{enumi})}
\renewcommand{\labelenumii}{(\roman{enumii})}     
        

\itemsep0pt
\parskip0pt

% A Lars Jonsson macro file, 2002-Sept-25
%%%%%%%%%%%%%%%%%%%%%%%%%%%%%%%%%%%%%%%%%
%
% Provides: \catchNameTitle         Takes name and title if used before 
%                                   \begin{document} and after \title \author
%
%           \timenow                returns a string with the time now.
%
%           \TODAY                  Creates a string of todays date
%
%           \MarkAll                access all headers in pagestyle{myheadings}
%
%           \NOW                    = \TODAY;\timenow
%
%           \NowFoot, \NowFootNum   creates footpage, w. filename, \NOW & page 
%
%           \DRAFT                  Write Draft in header
%
%           \fixNumberingInArticle  eq.num 4.12, same for fig.
%           \fixNumberingInAppendix eq.num B.3 and sec. number: A sec. title
%
%%%%%%%%%%%%%%%%%%%%%%%%%%%%%%%%%%%%%%%%%%%

\makeatletter 
% catches author and title for the headers, must be before \maketitle 
% that clears the 
% variables \@author and \@title

\def\catchNameTitle{
\global\let\Author\@author
\global\let\Title\@title
}

% creates a string with the time now.
\def\timenow{%
  \@tempcnta=\time \divide\@tempcnta by 60 \number\@tempcnta:\multiply
  \@tempcnta by 60 \@tempcntb=\time \advance\@tempcntb by -\@tempcnta
  \ifnum\@tempcntb <10 0\number\@tempcntb\else\number\@tempcntb\fi}

% creates a string with todays date Swedish version  
\def\TODAY{\number\year-\ifcase\month\or 01\or 02\or 03\or 04\or 05\or
    06\or 07\or 08\or 09\or 10\or 11\or 12\fi-\number\day}

% substitute for markboth or simular commands sets the page with #1 
%to #4 one in each corner on every page

\newcommand{\MarkAll}[4]{
\renewcommand{\ps@myheadings}{
\renewcommand{\@oddhead}{{\scriptsize #1 \hfil #2}}
\renewcommand{\@evenhead}{\@oddhead}
\renewcommand{\@evenfoot}{{\scriptsize #3}\hfil
	\textrm{\thepage}\hfil{\scriptsize #4}}
\renewcommand{\@oddfoot}{\@evenfoot}}
}

\def\NOW{\TODAY;\timenow}

\newcommand{\NowFootNum}{
\renewcommand{\@evenfoot}{{\scriptsize \jobname.tex}\hfil
	\textrm{\thepage}\hfil {\scriptsize \NOW}}
\renewcommand{\@oddfoot}{\@evenfoot}
}

\newcommand{\DRAFT}{
\renewcommand{\@evenhead}{\hfil DRAFT \hfil}
\renewcommand{\@oddhead}{\@evenhead}
}


\newcommand{\NowFoot}{
\renewcommand{\@evenfoot}{{\scriptsize \jobname.tex} {\scriptsize \NOW}}
\renewcommand{\@oddfoot}{\@evenfoot}
}


\def\fixNumberingInArticle{
\@addtoreset{figure}{section}
\@addtoreset{equation}{section}
\renewcommand{\thefigure}{\thesection.\arabic{figure}}  
\renewcommand{\theequation}{\thesection.\arabic{equation}}  
}

\def\fixNumberingInAppendix{
\setcounter{section}{0}
\renewcommand{\theequation}{\Alph{section}.\arabic{equation}}
\renewcommand{\thesection}{\Alph{section}}
\setcounter{equation}{0}
}


\makeatother





\pagestyle{fancy}
\fancyhead{} % clear all header fields
\renewcommand{\headrulewidth}{0pt} % no line in header area
\fancyfoot[L]{{\TODAY}  - \timenow} % other info in "inner" position of footer line