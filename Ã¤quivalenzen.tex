\documentclass[10pt, landscape]{article}

\usepackage{graphicx} % used for the Programma logo in the title.
\usepackage{datetime} % used to retrieve the current month and year.
\usepackage{titling} % used for creating the title.

\usepackage{amssymb} % used for mathematical symbols
\usepackage{amsmath} % used for math
\usepackage{amsfonts} % used for mathematical fonts

\usepackage{textcomp} % used to load fonts that contain special symbols.
\usepackage{lmodern} % used because it is supposed to look better.
\usepackage{tabu} % For sized, content wrapping tables.
\usepackage[T1]{fontenc} % used because it is recommended for PDF viewers being able to process strange symbols in the right manner.

\usepackage{color} % used for coloring.
\usepackage{fontspec}
\usepackage{courier} % used for something that I forgot.

\usepackage{rotating} % used to rotate large inference trees.

\usepackage{import} %for inputting images in the 'images' directory.

\usepackage{stmaryrd} %used for semantic brackets: \llbracket \rrbracket


% Macro to include the a pdf file that has been generated with inkscape from the file images/#1.tex
% #2 is display width.
\newcommand {\svg}[2]{\def\svgwidth{#2}
\vtop{%
  \vskip0pt
  \hbox{
    \import{images/}{#1.pdf_tex}
  }
}}

\def\labelitemii{$\circ$} %Makes the second level bullet sign a circle.
\renewcommand{\ttdefault}{lmtt}  % Select a latin modern font as the standard typewriter font!

% Determine character spacing:
\setlength{\parindent}{0 cm}
\setlength{\parskip}{0.2 cm}



%Macro for TODOs:
\newcommand{\todo}[1]{\textcolor{red}  {\textbf{TODO: #1}}}


% For the end of proofs:
\newcommand{\qed}{\hfill $\blacksquare$}


%This macro is used to split text into columns:
\newenvironment{column}[1]
{\begin{minipage}[t]{#1\linewidth}}{\end{minipage} \bigskip\bigskip}
\newcommand{\nextColumn}[1]{\end{minipage}\begin{minipage}[t]{#1\linewidth}}


% No idea what these do:
\usepackage[small,compact]{titlesec}
\RequirePackage{etoolbox}
\RequirePackage{environ}

% Some ornaments:
\newcommand{\pgbreakhere}{\vskip0pt plus3cm\penalty-250\vskip0pt plus-3cm}

\newcommand{\raisedrule}[2][0em]{\leavevmode\leaders\hbox{\rule[#1]{1pt}{#2}}\hfill\kern0pt}

\newcommand{\shiftup}{\vskip-12mm ~\\}

% Prints a nice document head, with given title.
\newcommand{\Head}[1]{
   \begin{center}
   				\renewcommand*\familydefault{\sfdefault}
   				\normalfont\LARGE 
                 Nebenl\"aufige Programmierung (SS 2013)\\[1ex]
                 \raisedrule[5pt]{1pt}\hspace{2em}
                 #1
                 \hspace{2em}\raisedrule[5pt]{1pt}\\[1ex]
   \end{center}
  \smallskip
}
      
\renewcommand{\labelenumi}{(\alph{enumi})}
\renewcommand{\labelenumii}{(\roman{enumii})}     
        

\itemsep0pt
\parskip0pt





\usepackage[top=2cm, bottom=3cm, left=2.5cm, right=2.5cm]{geometry} % page margins

\begin{document}

\Head{Äquivalenzen}

Wir kennen folgende Äquivalenzrelationen:

\newcommand{\ssubset}{\begin{sideways} $\subset$ \end{sideways}}
\newcommand{\ssupset}{\begin{sideways} $\supset$ \end{sideways}}
\newcommand{\noCongr}[1]{{\color{red}#1}}
\[\begin{array} {@{}ccccccccccc}
    &         &                      &         & \sim_{tr}  &         &        & \subset &   &         & \sim_{wtr}        \\
    &         &                      &         & \ssubset   &         &            & &       &         & \ssubset          \\
id  & \subset & \noCongr{\sim_{iso}} & \subset & \sim       & \subset &  \approx^+ &=& \cong & \subset & \noCongr{\approx} \\ 
    &         &                      &         & \ssupset   &         &            & &       &         & \ssupset          \\
    &         &                      &         & \sim_{ttr} &         &        & \subset &   &         & \sim_{wttr}       \\
\end{array}\]
Weitere Inklusionen gelten nicht. Alle, bis auf die \noCongr{gekennzeichneten} Relationen, sind Kongruenzrelationen!

\bigskip

\begin{column}{.35}

	\small
	\svg{trEq_neq_sBiSim}{6cm} 
	\vspace {2cm}
	\svg{tr_notIn_ttr}{6cm} 	
	
\hspace{1.5cm}	
\nextColumn{.35}

	\small
	\svg{iso_incongruent}{6cm}
	
\hspace{1.5cm}	
\nextColumn{.35}

	\svg{wBisim_incongruent}{5cm}
	
\end{column}

\end{document}
